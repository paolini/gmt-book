\IfFileExists{stacks-project.cls}{%
\documentclass{stacks-project}
}{%
\documentclass{amsart}
}

% For dealing with references we use the comment environment
\usepackage{verbatim}
\newenvironment{reference}{\comment}{\endcomment}
%\newenvironment{reference}{}{}
\newenvironment{slogan}{\comment}{\endcomment}
\newenvironment{history}{\comment}{\endcomment}

% For commutative diagrams we use Xy-pic
\usepackage[all]{xy}

% We use 2cell for 2-commutative diagrams.
\xyoption{2cell}
\UseAllTwocells

% We use multicol for the list of chapters between chapters
\usepackage{multicol}

% This is generall recommended for better output
\usepackage{lmodern}
\usepackage[T1]{fontenc}

% For cross-file-references
\usepackage{xr-hyper}

% Package for hypertext links:
\usepackage{hyperref}

% For any local file, say "hello.tex" you want to link to please
% use \externaldocument[hello-]{hello}
\externaldocument[introduction-]{introduction}
\externaldocument[conventions-]{conventions}
\externaldocument[sets-]{sets}
\externaldocument[fdl-]{fdl}
\externaldocument[index-]{index}

% Theorem environments.
%
\theoremstyle{plain}
\newtheorem{theorem}[subsection]{Theorem}
\newtheorem{proposition}[subsection]{Proposition}
\newtheorem{lemma}[subsection]{Lemma}

\theoremstyle{definition}
\newtheorem{definition}[subsection]{Definition}
\newtheorem{example}[subsection]{Example}
\newtheorem{exercise}[subsection]{Exercise}
\newtheorem{situation}[subsection]{Situation}

\theoremstyle{remark}
\newtheorem{remark}[subsection]{Remark}
\newtheorem{remarks}[subsection]{Remarks}

\numberwithin{equation}{subsection}

% Macros
%
\def\lim{\mathop{\mathrm{lim}}\nolimits}
\def\colim{\mathop{\mathrm{colim}}\nolimits}
\def\Spec{\mathop{\mathrm{Spec}}}
\def\Hom{\mathop{\mathrm{Hom}}\nolimits}
\def\Ext{\mathop{\mathrm{Ext}}\nolimits}
\def\SheafHom{\mathop{\mathcal{H}\!\mathit{om}}\nolimits}
\def\SheafExt{\mathop{\mathcal{E}\!\mathit{xt}}\nolimits}
\def\Sch{\mathit{Sch}}
\def\Mor{\mathop{\mathrm{Mor}}\nolimits}
\def\Ob{\mathop{\mathrm{Ob}}\nolimits}
\def\Sh{\mathop{\mathit{Sh}}\nolimits}
\def\NL{\mathop{N\!L}\nolimits}
\def\CH{\mathop{\mathrm{CH}}\nolimits}
\def\proetale{{pro\text{-}\acute{e}tale}}
\def\etale{{\acute{e}tale}}
\def\QCoh{\mathit{QCoh}}
\def\Ker{\mathop{\mathrm{Ker}}}
\def\Im{\mathop{\mathrm{Im}}}
\def\Coker{\mathop{\mathrm{Coker}}}
\def\Coim{\mathop{\mathrm{Coim}}}

% Boxtimes
%
\DeclareMathSymbol{\boxtimes}{\mathbin}{AMSa}{"02}

%
% Macros for moduli stacks/spaces
%
\def\QCohstack{\mathcal{QC}\!\mathit{oh}}
\def\Cohstack{\mathcal{C}\!\mathit{oh}}
\def\Spacesstack{\mathcal{S}\!\mathit{paces}}
\def\Quotfunctor{\mathrm{Quot}}
\def\Hilbfunctor{\mathrm{Hilb}}
\def\Curvesstack{\mathcal{C}\!\mathit{urves}}
\def\Polarizedstack{\mathcal{P}\!\mathit{olarized}}
\def\Complexesstack{\mathcal{C}\!\mathit{omplexes}}
% \Pic is the operator that assigns to X its picard group, usage \Pic(X)
% \Picardstack_{X/B} denotes the Picard stack of X over B
% \Picardfunctor_{X/B} denotes the Picard functor of X over B
\def\Pic{\mathop{\mathrm{Pic}}\nolimits}
\def\Picardstack{\mathcal{P}\!\mathit{ic}}
\def\Picardfunctor{\mathrm{Pic}}
\def\Deformationcategory{\mathcal{D}\!\mathit{ef}}


% OK, start here
%
\begin{document}

\title{Conventions}


\maketitle

\phantomsection
\label{section-phantom}

\tableofcontents

\section{Comments}
\label{section-comments}

\noindent
The philosophy behind the conventions used in writing these documents is
to choose those conventions that work.

\section{Set theory}
\label{section-sets}

\noindent
We use Zermelo-Fraenkel set theory with the axiom of choice.
See \cite{Kunen}. We do not use
universes (different from SGA4). We do not stress set-theoretic issues,
but we make sure everything is correct (of course) and so we do not ignore
them either.


\section{Categories}
\label{section-categories}

\noindent
A category $\mathcal{C}$ consists of a set of objects and, for each pair
of objects,
a set of morphisms between them. In other words, it is what is called
a ``small'' category in other texts. We will use ``big'' categories
(categories whose objects form a proper class)
as well, but only those that are listed in Categories,
Remark \ref{categories-remark-big-categories}.

\section{Algebra}
\label{section-algebra}

\noindent
In these notes a ring is a commutative ring with a $1$. Hence the
category of rings has an initial object $\mathbf{Z}$ and a final
object $\{0\}$ (this is the unique ring where $1 = 0$). Modules are
assumed unitary. See \cite{Eisenbud}.

\section{Notation}
\label{section-notation}

\noindent
The natural integers are elements of $\mathbf{N} = \{1, 2, 3, \ldots\}$.
The integers are elements of $\mathbf{Z} = \{\ldots, -2, -1, 0, 1, 2, \ldots\}$.
The field of rational numbers is denoted $\mathbf{Q}$.
The field of real numbers is denoted $\mathbf{R}$.
The field of complex numbers is denoted $\mathbf{C}$.

\begin{multicols}{2}[\section{Other chapters}]
\noindent
Preliminaries
\begin{enumerate}
\item \hyperref[introduction-section-phantom]{Introduction}
\item \hyperref[conventions-section-phantom]{Conventions}
\item \hyperref[sets-section-phantom]{Set Theory}
\end{enumerate}
\end{multicols}


\bibliography{my}
\bibliographystyle{amsalpha}

\end{document}
